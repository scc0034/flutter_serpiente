\apendice{Documentación de usuario}

\section{Introducción}
En este apartado se recoge todo lo que un usuario necesita conocer para poder ejecutar la aplicación en su teléfono móvil Android personal y los requisitos mínimos necesarios.

\section{Requisitos de usuarios}
Los requerimientos necesarios para poder ejecutar la aplicación en el teléfono son:
\begin{itemize}
	\tightlist
	\item
	Disponer de un terminal que al menos tenga la versión de Android en \emph{Lollipop}, ya que esta cuenta con el SDK mínimo con el que se ha desarrollado la aplicación, siendo este el 21~\cite{wiki:versionAndroid}.
	\item
	Es necesario disponer de conexión a Internet, tanto como para bajarse la aplicación en \emph{Store}, como para usar los servicios, ya que es necesario logearse con \emph{Google}
	\item La clasificación de contenido, es PEGI 3~\cite{wiki:pegi}. 
	\item En el caso de que la descarga se realice mediante la tienda oficial, los países para los que la aplicación está disponible son: España, Francia, Portugal e Irlanda. En el caso de que no sea así, será necesario descargar desde Github, como se explica en la página~\pageref{descargaGit}.
\end{itemize}

\section{Instalación}
La instalación se puede hacer a través de dos métodos diferentes:

\subsection{GitHub:}\label{descargaGit}
Desde el repositorio donde está el proyecto en el apartado de las \emph{releases}~\cite{github:releases}, podemos encontrar la última de las versiones compiladas, con el fin de descargarla.

Al ser una aplicación de orígenes desconocidos debemos permitir dando permisos de la siguiente manera:
\begin{enumerate}
	\tightlist
	\item Ir a los ajustes del terminal.
	\item Apartado de privacidad o seguridad.
	\item Activar el \emph{slider} de `Orígenes desconocidos''.
	\item Ejecutar el fichero .apk que acabamos de descargar.
	\item Instalar y abrir la aplicación.
\end{enumerate}

\subsection{Play Store:}
La manera más cómoda de hacerlo, es dirigirse \href{https://play.google.com/store/apps/details?id=com.ubu.flutter_snake}{Flutter games}, que es el enlace de descarga para dispositivos móviles Android. La versión disponible es la 6, ya que he tenido que realizar varias pruebas, con el fin de validar la disponibilidad, por eso el número que tiene.



\imagen{manual/tienda.jpg}{Tienda con el proyecto}

El proceso de instalación es simple, solo tenemos que dar al botón de instalar. En el caso de que nos encontremos en el navegador web, nos deja elegir el dispositivo que tenemos vinculado a nuestra cuenta de \emph{Google}. Si por otra parte nos encontramos desde el terminal, se instalará sin ningún problema.

En el caso de que se lancen nuevas versiones del producto, las actualizaciones se realizarán de forma automática.

\section{Manual del usuario}
Se pretende mostrar el funcionamiento de cada una de las ventanas que están disponibles en la aplicación, con el fin de poder informar a los usuarios del funcionamiento de cada una de estas.

Las ventanas de las que consta la aplicación son las siguientes:

\begin{itemize}
	\tightlist
	\item Log in.
	\item Home.
	\item Menú de los juegos.
	\item Snake.
	\item Ranking.
	\item Cuatro en raya menú.
	\item Cuatro en raya invitar.
	\item Cuatro en raya unirse.
	\item Cuatro en raya juego.
	\item About.
	\item Settings.
\end{itemize}

\subsection{Log in}
Es la primera ventana que nos encontramos cuando lanzamos la aplicación. Es necesario que nos registremos siempre con la cuenta de \emph{Google} que tengamos disponible, ya que muchos de los servicios están en la nube, no esta permitido el acceso mediante anonimato.

\begin{figure}[H]
	\centering
	\includegraphics[height=0.7\textwidth]{manual/loginpage.jpg}
	\caption{Log in page}\label{fig:loginpage}
\end{figure}

Cuando pulsamos en el botón de \emph{sign in with Google}, nos aparecerá la ventana siguiente~\ref{fig:formlog}, donde nos pedirá que ingresemos el usuario de nuestra cuenta de \emph{Google}. También se puede consultar la política de privacidad~\ref{fig:policy}

\begin{figure}[H]
	\centering
	\includegraphics[height=0.8\textwidth]{manual/loginform.jpg}
	\caption{Formulario sign in}\label{fig:formlog}
\end{figure}

\begin{figure}[H]
	\centering
	\includegraphics[height=0.8\textwidth]{manual/policy.jpg}
	\caption{Política de privacidad}\label{fig:policy}
\end{figure}

Una vez completemos este proceso, la ventana siguiente a la que nos redirige la aplicación es el Home~\ref{fig:homepage}.


\subsection{Home}
Esta página, es donde se muestra el póster del proyecto, además de tener el acceso al menú de la aplicación. Para acceder a este tenemos que deslizar lateralmente a la derecha, y para cerrarlo, tenemos que hacer el proceso inverso deslizando hacia la izquierda o pulsando fuera del menú~\ref{fig:homepage}.

\begin{figure}[H]
	\centering
	\includegraphics[height=0.8\textwidth]{manual/home.jpg}
	\caption{Home}\label{fig:homepage}
\end{figure}

Como podemos ver en la imagen~\ref{fig:homepage}, tenemos un \emph{banner} donde se nos muestra la publicidad, en el caso de que pinchemos en el, nos abrirá el navegador para mostrarnos más datos referentes al producto que se está anunciando. 

\begin{figure}[H]
	\centering
	\includegraphics[height=0.8\textwidth]{manual/menuhome.jpg}
	\caption{Home}\label{fig:menuhome}
\end{figure}
 
