\apendice{Especificación de Requisitos}

\section{Introducción}
En este apéndice se detallan los requisitos del proyecto, como los funcionales y no funcionales. La finalidad es hacer de intermediario entre el cliente y los programadores, con el objetivo de ayudar a entender, comprender y analizar la aplicación. 

\section{Objetivos generales}
Este trabajo final de grado focaliza la aplicación mediante dos puntos de vista totalmente diferentes, dependiendo de las situaciones futuras a las que se someta la aplicación, siempre englobado en el marco de que es un producto que nos encarga la universidad de Burgos, como cliente final.

\begin{itemize}
	\item Colección de videojuegos. Tener una gran cantidad de estos mismos, para lograr sacar un rendimiento económico gracias a lo publicidad, por lo que es necesario que la aplicación tenga una gran repercusión en el mercado. Esto implica que a futuro se debe de hacer una inversión en publicidad y \emph{marketing}.
	\item \emph{Porfolio} \cite{wiki:portafolio} o escaparate con el que mostrar las capacidades técnicas, herramientas usadas o lenguajes de programación aprendidos, ante los equipos de recursos humanos en las empresas, llegado el momento de tener que buscar trabajo. 
\end{itemize}

\section{Catalogo de requisitos}
A continuación se numeran los requisitos del proyecto extraídos de los generales.

\subsection{Requisitos funcionles}
Por cada uno de los niveles de requisitos funcionales, va a tener una correlación con la pantalla en que se muestra.

\begin{itemize}
\tightlist
	\item \textbf{RF-1 Ventana de \emph{sign in:}} La aplicación tiene que contener una ventana principal con la que los usuarios puedan iniciar la sesión, con el fin de tener servicios en la nube.
	
	\begin{itemize}
	\tightlist
	\item \textbf{RF-1.1 Botón de inicio sesión:} Mediante un elemento clickable se debe de tener acceso a un formulario para poder entrar mediante la cuenta de \emph{Google}.
	\item \textbf{RF-1.2 Consultar la política de privacidad:} Un enlace en el que se pueda mostrar todo lo referente a la política de privacidad.
	\item \textbf{RF-1.3 Cerrar sesión:} ser capaz de salir de la aplicación \emph{sign out}, desde cualquier ventana.
	\end{itemize}
	
	\item \textbf{RF-2 Menú de navegación:} Se considera necesario tener la capacidad de navegar de una parte a otra dentro de la aplicación, sin tener que pasar entre ventanas, mediante un menú. 
	
	\begin{itemize}
		\tightlist
		\item \textbf{RF-2.1 Menú lateral} el usuario debe poder entrar al menú mediante un \emph{scroll lateral}.
		\item \textbf{RF-1.2 Consultar la política de privacidad:} Un enlace en el que se pueda mostrar todo lo referente a la política de privacidad.
		\item \textbf{RF-1.3 Cerrar sesión:} ser capaz de salir de la aplicación \emph{sign out}, desde cualquier ventana.
	\end{itemize}
	
\end{itemize}


\section{Especificación de requisitos}


