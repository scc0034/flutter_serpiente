\apendice{Documentación técnica de programación}

\section{Introducción}
En este anexo se describe la documentación técnica de programación para este proyecto. Incluye los primeros pasos que son la instalación del proyecto, la estructura de la aplicación o finalmente como compilarlo, desplegarlo o los diferentes tipos de configuraciones realizados. La idea es poder facilitar a los futuros desarrolladores una guía con la que poder comenzar, en el caso de que quisieran continuar con el trabajo.

\section{Estructura de directorios}
El repositorio se encuentra alojado en \href{https://github.com/scc0034/flutter_serpiente}{Github}. La estructura de ficheros que sigue es la siguiente:

\begin{itemize}
\item \textbf{./}
Directorio raíz del que cuelgan todas los demás ficheros. Este contiene uno de los archivos más importantes, que es es \emph{pubspec.yalm}. Este archivos se usa para hacer las importaciones de lós paquetes con las funcionalidades que queramos dar a nuestra aplicación.

\begin{itemize}
	\item \textbf{build}: Este directorio contiene todo lo relativo a las compilaciones, es decir, tanto como para hacer las pruebas en local de la aplicación , o crear los \emph{releases} que creamos oportunos. Además contiene todo lo relativo a las conexiones con Android Studio y Firebase, ya que necesita hacer las llamadas a este para lanzar los emuladores con la máquina virtual correspondiente. Dentro de esta estructura algunos de los ficheros más importantes son:
	\begin{itemize}
		\item \textbf{key.properties}: propeidades de la key, ya que esta nos permite desplegar la aplicación en la \emph{Play Store}. Es algo que no se tiene que perder ni modificar, ya que es de sumo valor. 
		\item \textbf{app/google-services.json}: fichero que descargamos desde Firebase, para que la aplicación tenga las conexiones con este \emph{Cloud service}, es decir, contienen las claves de conexión. En el caso de que tengamos que lanzar la app con otro de servicio de Firebase, podemos hacerlo cambiando este fichero.
		\imagen{techprog/google-services.jpg}{Google services}
		\item \textbf{app/build.gradle}: Fichero que contiene lo necesario para hacer las compilaciones, ya que como vemos tiene el SDK mínimo y máximo con el que trabaja (limitando el número de dispositivos que son compatibles),el número de versión, ya que cuando lo subamos a la \emph{Play Store}, es algo que debemos de revisar, ya que si no vamos a tener problemas de versionado. Además de los parámetros usados en la clave como podemos ver en la siguiente imagen.
		\imagen{techprog/app_build_gradle.jpg}{app/build.gradle}
		\item \textbf{app/keysnake.jks}: clave cifrada generada mediante el comando \ref{fig:commandkey}, esta no se puede perder, ya que sin ella es imposible desplegar la aplicación en la \emph{Play Store}.Es conveniente hacer alguna copia de seguridad en local.
		
		\begin{figure}
			\centering
			\includegraphics[width=1.15\textwidth]{/techprog/keysnakecommand}
			\caption{Comando generar clave key}
			\label{fig:commandkey}
		\end{figure}	
		\item \textbf{c}: 
	\end{itemize}
	\item \textbf{b}: 
	\item \textbf{c}: 
\end{itemize}

Tambien contiene los siguientes archivos:
\begin{itemize}
	\item \textbf{a}: 
	\item \textbf{b}:
	\item \textbf{c}:
\end{itemize}


\end{itemize}

\section{Manual del programador}

\section{Compilación, instalación y ejecución del proyecto}

\section{Pruebas del sistema}
