\capitulo{3}{Conceptos teóricos}

La parte del proyecto más importante es el proceso de lanzar una aplicación Android desde cero, para ello se ha tenido que realizar una investigación para determinar que herramientas, lenguajes de programación, entornos de desarrollo son los más apropiados para un despliegue ágil.

Todo esto se sitúa en un mercado altamente competitivo, con una gran variabilidad (usuarios, empresas, desarrolladores ...) y con un constante cambio. 

\section{Frameworks}
Un framework (de origen anglosajón, marco de trabajo), según lo que dice la wikipedia~\cite{wiki:framework}, es una estructura conceptual y tecnológica de soporte definido, normalmente por módulos de software concretos, que sirve de base para la organización y el desarrollo de software. Las ventajas que ofrece son varias, entre las que se puede destacar:

\begin{itemize}
	\item \textbf{Evita repetición de código:} las partes más usadas, pasan a ser algo del \emph{core} del framework.
	\item \textbf{Uso de buenas prácticas:} muchos de estos están basados en patrones de diseño que nos obligan a usar.
	\item \textbf{Elementos avanzados integrados:} cosas complejas y que implementarlas llevaría mucho tiempo, suelen venir integradas.
	\item \textbf{Desarrollo ágil:} por los factores anteriores, podemos centrarnos más en la lógica de negocio de la aplicación que se desea hacer, de una manera más rápida y segura.
\end{itemize}

Por lo tanto, es necesario trabajar mediante frameworks, ya que nos garantizar una aplicación de mayor calidad, que si la hacemos desde cero, en código nativo.

\subsection{Opciones disponibles}
En el mercado tenemos muchos frameworks disponibles, por lo que elegir uno no es tarea sencilla, ya que como veremos cada uno tiene sus pros y contras




\section{Imágenes}

Se pueden incluir imágenes con los comandos standard de \LaTeX, pero esta plantilla dispone de comandos propios como por ejemplo el siguiente:

\imagen{escudoInfor}{Autómata para una expresión vacía}



\section{Listas de items}

Existen tres posibilidades:

\begin{itemize}
	\item primer item.
	\item segundo item.
\end{itemize}

\begin{enumerate}
	\item primer item.
	\item segundo item.
\end{enumerate}

\begin{description}
	\item[Primer item] más información sobre el primer item.
	\item[Segundo item] más información sobre el segundo item.
\end{description}
	
\begin{itemize}
\item 
\end{itemize}

\section{Tablas}

Igualmente se pueden usar los comandos específicos de \LaTeX o bien usar alguno de los comandos de la plantilla.

\tablaSmall{Herramientas y tecnologías utilizadas en cada parte del proyecto}{l c c c c}{herramientasportipodeuso}
{ \multicolumn{1}{l}{Herramientas} & App AngularJS & API REST & BD & Memoria \\}{ 
HTML5 & X & & &\\
CSS3 & X & & &\\
BOOTSTRAP & X & & &\\
JavaScript & X & & &\\
AngularJS & X & & &\\
Bower & X & & &\\
PHP & & X & &\\
Karma + Jasmine & X & & &\\
Slim framework & & X & &\\
Idiorm & & X & &\\
Composer & & X & &\\
JSON & X & X & &\\
PhpStorm & X & X & &\\
MySQL & & & X &\\
PhpMyAdmin & & & X &\\
Git + BitBucket & X & X & X & X\\
Mik\TeX{} & & & & X\\
\TeX{}Maker & & & & X\\
Astah & & & & X\\
Balsamiq Mockups & X & & &\\
VersionOne & X & X & X & X\\
} 
